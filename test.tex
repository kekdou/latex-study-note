\documentclass{article}

\usepackage{amsmath}
\usepackage{graphicx}
\usepackage{ctex}
\usepackage{multirow}
\usepackage{booktabs}

\title{Test Latex \\ \large simple test}
\author{kekdou \thanks{THU} \and
kaf \thanks{神椿}}
\date{\today}

\begin{document}
\maketitle

\begin{abstract}
this is a simple test of latex.
\end{abstract}

\noindent \textbf{关键词:} latex, test

\section{引言} \label{sec:intro}
I only need this article to test some latex functions.

\section{功能} \label{sec:functions}
\subsection{文本样式} \label{sec:textstyle}
我可以使用\textbf{加粗},\textit{斜体},和\underline{下划线}。
我还可以使用不同大小的字体,如:
\begin{itemize}
\item \tiny{极小号字体}
\item \scriptsize{小脚号字体}
\item \footnotesize{脚注号字体}
\item \small{小号字体}
\item \normalsize{正常字体}
\item \large{大号字体}
\item \Large{更大号字体}
\item \LARGE{特大号字体}
\item \huge{超级大号字体}
\end{itemize}

\subsection{空格} \label{sec:space}
空          格 \\
空
格 \\
空\quad 格 \\
空\qquad 格

\subsection{换行} \label{sec:newline}
这是第一行。\\
这是第二行。

这是新一段的第一行,不信看看缩进。

\subsection{列表} \label{sec:list}
无序列表:
\begin{itemize}
\item apple
\item banana
\item orange
\end{itemize}

\begin{itemize}
\item apple
\item banana
\begin{itemize}
\item tomato
\item potato
\end{itemize}
\item orange
\end{itemize}

有序列表:
\begin{enumerate}
\item first step
\item second step
\item third step
\end{enumerate}

\begin{enumerate}
\item first step
\item second step
\begin{enumerate}
\item first step
\item second step
\end{enumerate}
\item third step
\end{enumerate}

\subsection{数学公式} \label{sec:math}
行内公式: \\
牛顿第二定律是 $F=ma$

独立公式: \\
%这是一个没有编号的独立公式:
\[
x = \frac{-b \pm \sqrt{b^2-4ac}}{2a}
\]
%这是一个有编号的独立公式:
\begin{equation}
y = ax^2 + bx + c
\end{equation}
关于多行公式的对齐问题:
\begin{align}
y&=(x+1)^2 \\      % \\用于换行
 &=x^2 + 2x + 1    % &指定对齐点
\end{align}

\[
\begin{pmatrix}
a&b \\
c&d
\end{pmatrix}
\]

\[
f(x) =
\begin{cases}
x^2+1 & \text{ if } x>0 \\
0 & \text{ if } x = 0 \\
x-1 & \text{ if } x<0
\end{cases}
\]

\subsection{表格} \label{sec:table}
这是一个段落,紧接着一个默认对齐的表格:
\begin{tabular}{|l|l|}
\hline
第一行&文本 \\
\hline
第二行&文本 \\
\hline
\end{tabular}
表格后的文本。

\vspace{1cm}  %添加一些垂直间距

这是一个段落,紧接着一个顶端对齐的表格:
\begin{tabular}[t]{|l|l|}
\hline
第一行&文本 \\
\hline
第二行&文本 \\
\hline
\end{tabular}
表格后的文本。

\vspace{1cm}   %添加一些垂直间距

这是一个段落,紧接着一个底端对齐的表格:
\begin{tabular}[b]{|l|l|}
\hline
第一行&文本 \\
\hline
第二行&文本 \\
\hline
\end{tabular}
表格后的文本。

\vspace{1cm}   %添加一些垂直间距

\begin{tabular}{|c|lll|p{4em}|}  
\hline
姓名&地理&生物&化学&备注 \\ 
\hline
小红&90&85&92&优秀 \\ 
\hline
张华&87&75&79&良好 \\ 
\hline
大明&91&75&97&存在偏科现象 \\ 
\hline
\end{tabular}

\vspace{1cm}   %添加一些垂直间距

\begin{tabular}{l @{--} r}
  起始时间 & 结束时间 \\
  10:00 & 11:00 \\
  12:30 & 13:00 \\
\end{tabular}

\vspace{1cm}   %添加一些垂直间距

\begin{tabular}{|c|c|c|}
\hline
\multicolumn{2}{|c|}{合并两列}&列3 \\
\hline
A&B&C \\
\hline
\end{tabular}

\vspace{1cm}   %添加一些垂直间距

\begin{tabular}{|c|c|c|}
\hline
\multirow{2}{*}{合并两行}&A&B \\ \cline{2-3}  %只画部分横线
& C & D \\
\hline
\end{tabular}

\vspace{1cm}   %添加一些垂直间距

\begin{tabular}{|c|c|c|}
\hline
\multicolumn{2}{|c|}{\multirow{2}{*}{合并行和列}}&列3  \\ \cline{3-3}
\multicolumn{2}{|c|}{}&列3 \\
\hline
A&B&C \\
\hline
\end{tabular}

\vspace{1cm}   %添加一些垂直间距

\begin{tabular}{|c|}
\hline
红 \\ \hline
红 \vline 黄 \\ \hline
红 \vline 黄 \vline 蓝 \\ \hline
\end{tabular} 

\vspace{1cm}   %添加一些垂直间距

\begin{tabular}{|c|}
\hline
红 \\ 
\hline
\begin{tabular}{@{}c|c@{}}
红&蓝 
\end{tabular} \\
\hline
\begin{tabular}{@{}c|c|c@{}}
红&蓝&黄
\end{tabular} \\
\hline
\end{tabular}

\vspace{1cm}   %添加一些垂直间距

\begin{table}[h]
\centering
\caption{水果价格表}
\label{tab:price}
\begin{tabular}{|l|c|r|}
\hline
水果&数量&价格 \\
\hline
苹果&5&10 \\
\hline
香蕉&3&6 \\
\hline
\end{tabular}
\end{table}

\vspace{1cm}   %添加一些垂直间距

\begin{table}[htbp]
\centering
\caption{三线表示例}
\label{tab:example}
\begin{tabular}{lcr}  %列中没有竖线
\toprule   %顶线
项目&数量&价格 \\
\midrule   %中线
苹果&5&10元 \\
香蕉&3&6元 \\
橙子&7&14元 \\
\bottomrule % 底线
\end{tabular}
\end{table}

\subsection{图片} \label{sec:figure}
\begin{figure}[htbp]
\centering
\includegraphics[width=0.8\textwidth]{kaf.jpg}
\caption{示例图片}
\label{fig:flowchart}
\end{figure}

章节\ref{sec:math} \\
章节\ref{sec:figure} \\
\end{document}